\documentclass[12pt, a4paper]{article}

%-----------------------------------------------------------------------
% Preamble by the organize commitee

\usepackage[utf8]{inputenc}
\usepackage[english]{babel}
\usepackage[margin=2.5cm]{geometry}
\usepackage{setspace}
\usepackage{indentfirst}
\usepackage{graphicx}
\usepackage{xcolor}
\usepackage{fancyhdr}
\usepackage{url}
\usepackage{enumerate}
\usepackage{amsmath, amsthm, amsfonts, amssymb, amsxtra}
\usepackage{bm}

\pagestyle{fancy}
\fancyhf{}
\lhead{$63^{\textrm{a}}$ RBras}
\rhead{May 23 to 25, 2018, Curitiba - PR}
% \cfoot{\thepage}
\renewcommand{\headrulewidth}{0.4pt}
\addtolength{\headheight}{12.0pt}

\makeatletter
\def\@xfootnote[#1]{%
  \protected@xdef\@thefnmark{#1}%
  \@footnotemark\@footnotetext}
\makeatother

\usepackage{hyperref}
\definecolor{mycol}{rgb}{0.0, 0.0, 0.5}
\urlstyle{tt}
\makeatletter
\hypersetup{
  pdftitle={\@title},
  pdfauthor={\@author},
  colorlinks=true,
  linkcolor=mycol,
  citecolor=mycol,
  filecolor=mycol,
  urlcolor=mycol,
  bookmarksdepth=4
}
\makeatother

%-----------------------------------------------------------------------
% Init the document

\begin{document}
\onehalfspacing

%-------------------------------------------
% Título
\begin{center}
  \textbf{
    \Large{Double COM-Poisson models: modelling mean and dispersion in
      the analysis of count data.}
  } \\[1em]
\end{center}

%-------------------------------------------
% Autores
\begin{flushright}
  {\bf Eduardo Elias Ribeiro Junior}
  \footnote[$\dagger$]{Contato:
    \href{mailto:jreduardo@usp.br}{\tt jreduardo@usp.br}}
  \footnote[1]{Department of Exact Sciences (LCE) - ESALQ-USP}
  \footnote[2]{Statistics and Geoinformation Laboratory (LEG) -
    UFPR}\\
  {\bf Clarice Garcia Borges Demétrio} \footnotemark[1]
\end{flushright}

\vspace*{0.5cm}

\noindent In this paper, we propose the double COM-Poisson models where
the mean and dispersion parameters of the COM-Poisson distribution are
modeled in terms of covariates. These model are quite flexible and allow
us to identify significant effects in both mean and dispersion of the
observed count data.\\

\noindent{\bf Palavras-chave}:
{\it COM-Poisson distribution, Varying dispersion,
     Double COM-Poisson models}.\\

\end{document}